\chapter{Ensayos y Mediciones}

\section{Motores y L293D}

Con el objetivo de observar el funcionamiento del L293D, los motores y
verificar si estos eran capaces de mover el vehículo con todos los
componentes a una velocidad aceptable, se realizó un prototipo con un Arduino, una caja de cartón que se utilizó 
como estructura provisoria (ver figura \ref{fig:ensayo1} y \ref{fig:ensayo2}) y el shield que contiene dos L293D, configurando el PWM con un ciclo de trabajo del 100\%. 
Para facilitar el prototipado, se utilizó la librería para control de motores desarrollada por Adafruit. El código de prueba para el Arduino es el siguiente:

\begin{lstlisting}
#include <AFMotor.h>

AF_DCMotor motor1(1);                   //Declara los motores a utilizar
AF_DCMotor motor2(2);
AF_DCMotor motor3(3);
AF_DCMotor motor4(4);

void setup() {
  Serial.begin(9600);                   //Inicializa la terminal serie a 9600 bps
  Serial.println("Prueba de motores");

  motor1.run(RELEASE);                  //Desbloquea el motor
  motor1.run(FORWARD);                  //Mueve para adelante
  motor1.setSpeed(255);                 //Velocidad entre 0 y 255, configurada al máximo

  motor2.run(RELEASE);
  motor2.run(FORWARD);
  motor2.setSpeed(255);

  motor3.run(RELEASE);
  motor3.run(FORWARD);
  motor3.setSpeed(255);

  motor4.run(RELEASE);
  motor4.run(FORWARD);
  motor4.setSpeed(255);
}

void loop() {
}
\end{lstlisting}

\paragraph{Conclusiones}

Los motores son lo suficientemente potentes como para mover el vehículo
a una velocidad considerable, siempre y cuando los L293D sean
alimentados con aproximadamente 8V, ya que hay una diferencia entre la
entrada del L293D y la salida a cada uno de los motores de
aproxidamente 2V (los motores van a su máxima velocidad con 6V).

\section{Sensor HC-SR04}

Una vez ensamblado el auto en su correspondiente chasis en conjunto con
la totalidad de sus componentes, realizamos las pruebas de detección de
objetos. Estas pruebas consistieron en direccionar el auto hacia un
obstáculo, contemplando el margen de frenado requerido para que el
vehículo no choque contra dicho obstáculo, realizando sucesivas
modificaciones en las variables de distancia de frenado hasta dar con la
indicada. El código en cuestión se muestra a continuación:

\begin{lstlisting}
void evaluar_colisiones() {

    // obtenemos la distancia actual
    distancia = ultrasonicSensorGetDistance(ULTRASONIC_SENSOR_0, CM);

    // antes de frenar por un obstaculo, se activa el buzzer para indicar la presencia del mismo
    if(distancia <= 100 && auto_avanzando()){
        sonido_colision();
    }else{
        desactivar_buzzer();
    }

    //Activamos el led de estado cuando detectamos una colision
    if(distancia <= 30){
        activar_led_colision();
        bloquear();
    }else{
        desactivar_led_colision();
        desbloquear();
    }


}
\end{lstlisting}

\section{Control remoto}

Por simplicidad, se utilizó una aplicación disponible en el play store
de Android, llamada BLEJoystick para manejar el vehículo vía el
protocolo BLE y el módulo HM-10. Se probó el funcionamiento del
Bluetooth utilizando el programa de ejemplo que viene en la librería
SAPI para el uso del módulo HM-10, y el programa Hercules, que permite
visualizar todos los comandos enviados desde la aplicación. Luego de
configurar el Hercules en modo serie, y encontrar el puerto COM donde
estaba conectada la EDU-CIAA, pudimos ver que la aplicación funcionaba
correctamente, enviando el caracter correspondiente según la tecla
presionada.

\begin{lstlisting}
// FUNCION PRINCIPAL, PUNTO DE ENTRADA AL PROGRAMA LUEGO DE ENCENDIDO O RESET.
int main( void )
{
   // ---------- CONFIGURACIONES ------------------------------

   // Inicializar y configurar la plataforma
   boardConfig();

   // Inicializar UART\_USB para conectar a la PC
   uartConfig( UART\_PC, 9600 );
   uartWriteString( UART_PC, "UART_PC configurada.\r\n" );

   // Inicializar UART\_232 para conectar al modulo bluetooth
   uartConfig( UART_BLUETOOTH, 9600 );
   uartWriteString( UART_PC, "UART_BLUETOOTH para modulo Bluetooth configurada.\r\n" );
   
   uint8_t data = 0;
   
   uartWriteString( UART_PC, "Testeto si el modulo esta conectado enviando: AT\r\n" );
   if( hm10bleTest( UART_BLUETOOTH ) ){
      uartWriteString( UART_PC, "Modulo conectado correctamente.\r\n" );
   }
   else{
      uartWriteString( UART_PC, "No funciona.\r\n" );
   }

   // ---------- REPETIR POR SIEMPRE --------------------------
   while( TRUE ) {

      // Si leo un dato de una UART lo envio a al otra (bridge)
      if( uartReadByte( UART_PC, &data ) ) {
         uartWriteByte( UART_BLUETOOTH, data );
      }
      if( uartReadByte( UART_BLUETOOTH, &data ) ) {
         if( data == 'h' ) {
            gpioWrite( LEDB, ON );
         }
         if( data == 'l' ) {
            gpioWrite( LEDB, OFF );
         }
         uartWriteByte( UART_PC, data );
      }
      
      // Si presiono TEC1 imprime la lista de comandos AT
      if( !gpioRead( TEC1 ) ) {
         hm10blePrintATCommands( UART_BLUETOOTH );
         delay(500);
      }
      
      // Si presiono TEC3 enciende el led de la pantalla de la app
      if( !gpioRead( TEC3 ) ) {
         uartWriteString( UART_BLUETOOTH, "LED_ON\r\n" );
         delay(500);
      }
      // Si presiono TEC4 apaga el led de la pantalla de la app
      if( !gpioRead( TEC4 ) ) {
         uartWriteString( UART_BLUETOOTH, "LED_OFF\r\n" );
         delay(500);
      }
   }

   // NO DEBE LLEGAR NUNCA AQUI, debido a que a este programa se ejecuta
   // directamenteno sobre un microcontroladore y no es llamado por ningun
   // Sistema Operativo, como en el caso de un programa para PC.
   return 0;
}
\end{lstlisting}

\section{Buzzer}

Al igual que con el sensor HC-SR04 los ensayos se realizaron con el auto
una vez ensamblado. Determinamos la necesidad de utilizar dos tipos de
intermitencias para diferenciar la detección de un obstáculo con la
reversa de vehículo. La practica de dicho ensayo no implicó ningún
inconveniente ya que solo consintió en modificar el código a gusto de
manera que los tiempos sean los buscados.

\begin{lstlisting}
void actualizar_buzzer(){
    if(tipo_de_sonido==1){
        contador++;
        if(contador==2){
            estado_buzzer=!estado_buzzer;
            gpioWrite(GPIO6, estado_buzzer);
            contador=0;
        }
    }

    if(tipo_de_sonido==2){
        contador++;
        if(contador==5){
            estado_buzzer=!estado_buzzer;
            gpioWrite(GPIO6, estado_buzzer);
            contador=0;
        }
    }
\end{lstlisting}

\section{Baterías de litio}

En base a mediciones realizadas previamente en la etapa de protipado,
encontramos conveniente utilizar dos baterías de litio, cuya capacidad
máxima de carga individual fue de 4v, lo cual a partir de su conexión en
serie nos permitió obtener los 8v requeridos para la alimentación de los
L293D así como también alimentar la EDU-CIAA con 5v gracias a la fuente
step-down.

\section{Fuente step down}

Una vez corregida la disposición de la fuente step-down, se procedió a
calibrarla, desconectando el resto de los componentes. Midiendo la
tensión de salida con un multímetro, se llevó la tensión de salida a 5v,
lo necesario para alimentar la EDU-CIAA.

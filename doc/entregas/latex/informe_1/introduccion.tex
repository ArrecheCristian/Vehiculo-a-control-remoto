

\chapter{Introducción}

\paragraph{}

Una de las principales aplicaciones que se le ha dado a la electrotecnia es el control de motores. En el ámbito de la electrónica se hace uso de pequeños motores con poca potencia que funcionan con corriente continua, como por ejemplo para mover autos a escala.

\paragraph{}

En los últimos años, con la aparición de plataformas de desarrollo como Arduino, se han desarrollado autos a control remoto artesanales; que proveen una gran diversidad de resultados debido a la plataforma abierta y variedad de componentes electrónicos.

\paragraph{}

La motivación del proyecto es obtener un producto que cuente con cierto nivel de complejidad y, al mismo tiempo, que sea visualmente gratificante. Consideramos que el control de un vehículo a control remoto conlleva una parte importante de investigación y desarrollo; y brinda un resultado que es tangible.

\paragraph{}

El proyecto es un vehículo móvil, equipado este con motores eléctricos que reaccionan dependiendo de las señales enviadas desde un mando inalámbrico del que depende el movimiento del automóvil. Además, colocaremos una serie de sensores para agregar mayor funcionalidad al proyecto reaccionando ante distintos eventos.

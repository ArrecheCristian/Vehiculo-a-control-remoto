\chapter{Requerimientos}

\vspace{12pt}
Aquellos requerimientos indicados con \textbf{OPT} son completamente opcionales.

\section{Requerimientos Funcionales}

\subsection{Requerimientos de Software}

\subsubsection{Control de Motores}

\begin{enumerate}
	\item El sistema debe accionar 4 motores individualmente, permitiendo girar en ambas direcciones
	\item El sistema debe accionar motores en conjunto, permitiendo avanzar, retroceder y girar
	\item El sistema debe permitir el bloqueo de movimiento en una dirección (por ejemplo debido a un obstáculo detectado por el sensor de proximidad)
	\item \textbf{OPT} El sistema debe indicar mediante una serie de LEDs el estado de movimiento del vehículo
\end{enumerate}

\subsubsection{Control Remoto}

\begin{enumerate}
	\item El sistema debe  establecer una conexión bluetooth con el sistema de control remoto
	\item El sistema debe leer las indicaciones del control remoto y ejecutar las acciones correspondientes
	\item El sistema debe indicar mediante un LED el estado de la conexión bluetooth
\end{enumerate}

\subsubsection{\textbf{OPT} Medición Sensores }

\begin{enumerate}
	\item Ante ausencia de luz externa, se deben encender las luces del vehículo
	\item Ante la presencia de obstáculos, se debe frenar el desplazamiento en esa dirección
	
\end{enumerate}

\subsection{Requerimientos de Hardware}

\subsubsection{Control de Motores}

\begin{enumerate}
	\item El vehículo debe utilizar puentes H para motores de corriente continua
	\item \textbf{OPT} El vehículo debe Indicar con un LED para cada motor si este está en movimiento
\end{enumerate}

\subsubsection{Baterías y cargador}

\begin{enumerate}
	\item El vehículo debe utilizar baterías de Litio para lograr la autonomía del vehículo
	\item El vehículo debe indicar mediante un LED cuando haya baja tensión
	\item El vehículo debe permitir cargar las baterías mediante un puerto USB
\end{enumerate}

\subsubsection{Componentes}

\begin{enumerate}
	\item Se debe contar con una llave de encendido/apagado y un LED que determine su estado
	\item \textbf{OPT} El sistema debe poseer un buzzer para indicar eventos o fallas
\end{enumerate}

\section{Requerimientos No Funcionales}

\begin{itemize}
	\item Utilización de la placa de desarrollo EDU-CIAA
	\item Estructura sólida y liviana, donde se colocan los motores y la EDU-CIAA
	\item El conexionado a los motores debe ser a través de borneras para poder cambiar la estructura.
	\item Programación en lenguaje C
	\item Desarrollo de PCB en formato de poncho
	\item Mecanismo de control simple
	\item El sistema debe manejar respuestas en tiempo real.
	\item Fecha de finalización y entrega del proyecto el día 16/12
	
\end{itemize}
